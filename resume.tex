%%%%%%%%%%%%%%%%%
% This is an example CV created using altacv.cls (v1.6.4, 13 Nov 2021) written by
% LianTze Lim (liantze@gmail.com), based on the
% Cv created by BusinessInsider at http://www.businessinsider.my/a-sample-resume-for-marissa-mayer-2016-7/?r=US&IR=T
%
%% It may be distributed and/or modified under the
%% conditions of the LaTeX Project Public License, either version 1.3
%% of this license or (at your option) any later version.
%% The latest version of this license is in
%%    http://www.latex-project.org/lppl.txt
%% and version 1.3 or later is part of all distributions of LaTeX
%% version 2003/12/01 or later.
%%%%%%%%%%%%%%%%

%% Use the "normalphoto" option if you want a normal photo instead of cropped to a circle
% \documentclass[10pt,a4paper,normalphoto]{altacv}

\documentclass[10pt,a4paper,ragged2e,withhyper]{altacv}




%% AltaCV uses the fontawesome5 package.
%% See http://texdoc.net/pkg/fontawesome5 for full list of symbols.

% Change the page layout if you need to
\geometry{left=1.25cm,right=1.25cm,top=1.5cm,bottom=1.5cm,columnsep=1.2cm}

% The paracol package lets you typeset columns of text in parallel
\usepackage{paracol}
\usepackage{graphicx}
\usepackage{fontawesome5}


\renewcommand{\cvevent}[4]{%
  \textbf{#1} % Job title
  \hfill % This will push the right minipage to the right
  \begin{minipage}[t]{.5\linewidth}
    \raggedleft % Aligns content to the right
    \small#3 % Location
    \\ % New line for the date right below location
    #4 % Date
  \end{minipage}
  \vspace{\baselineskip} % Adds some space after the event
}





\newlength{\mylen}

\makeatletter
% Modify the cvsection command to accept an icon name as the first argument
\RenewDocumentCommand{\cvsection}{o m}{%
  % Check if an icon name is provided
  \IfValueTF{#1}{%
    % If an icon is provided, measure the width including the icon
    \settowidth{\mylen}{\Large\bfseries\color{heading}#1\hspace{0.0em}#2}
    % Add space before the section title
    \vspace{1.5ex}
    % Print the section title with icon
    {\Large\bfseries\color{heading}#1\hspace{0.2em}#2}\\[-1ex]
  }{%
    % If no icon is provided, proceed as before
    \settowidth{\mylen}{\Large\bfseries\color{heading}#2}
    % Add space before the section title
    \vspace{1.5ex}
    % Print the section title without icon
    {\Large\bfseries\color{heading}#2}\\[-1ex]
  }
  % Center the rule: adjust the length here if you want to change the thickness
  \makebox[\mylen]{\rule{\mylen}{0.0pt}} % Use the measured width for the rule
  \vspace{2ex} % Adjusted space after the title underline
}
\makeatother

% Change the font if you want to, depending on whether
% you're using pdflatex or xelatex/lualatex
\ifxetexorluatex
  % If using xelatex or lualatex:
  \setmainfont{Lato}
\else
  % If using pdflatex:
  \usepackage[default]{lato}
\fi

% Change the colours if you want to
\definecolor{VividPurple}{HTML}{3E0097}
\definecolor{SlateGrey}{HTML}{2E2E2E}
\definecolor{LightGrey}{HTML}{666666}
% \colorlet{name}{black}
% \colorlet{tagline}{PastelRed}
\colorlet{heading}{VividPurple}
\colorlet{headingrule}{VividPurple}
% \colorlet{subheading}{PastelRed}
\colorlet{accent}{VividPurple}
\colorlet{emphasis}{SlateGrey}
\colorlet{body}{LightGrey}

% Change some fonts, if necessary
% \renewcommand{\namefont}{\Huge\rmfamily\bfseries}
% \renewcommand{\personalinfofont}{\footnotesize}
% \renewcommand{\cvsectionfont}{\LARGE\rmfamily\bfseries}
% \renewcommand{\cvsubsectionfont}{\large\bfseries}

% Define the length outside the command, to ensure it's only defined once






% Change the bullets for itemize and rating marker
% for \cvskill if you want to
\renewcommand{\itemmarker}{{\small\textbullet}}
\renewcommand{\ratingmarker}{\faCircle}

%% Use (and optionally edit if necessary) this .tex if you
%% want to use an author-year reference style like APA(6)
%% for your publication list
% Custom simplified bibliography for resume - shows title, year, publication only
\usepackage[backend=biber,style=apa6,sorting=ydnt]{biblatex}
\defbibheading{pubtype}{\cvsubsection{#1}}
\renewcommand{\bibsetup}{\vspace*{-\baselineskip}}
\AtEveryBibitem{\makebox[\bibhang][l]{\itemmarker}}
\setlength{\bibitemsep}{0.25\baselineskip}
\setlength{\bibhang}{1.25em}

% Hide author names
\AtEveryBibitem{%
  \clearname{author}%
  \clearname{editor}%
  \clearname{translator}%
}

% Hide unnecessary fields for cleaner display
\AtEveryBibitem{%
  \clearfield{doi}%
  \clearfield{isbn}%
  \clearfield{issn}%
  \clearfield{url}%
  \clearfield{urldate}%
  \clearfield{eprint}%
  \clearfield{eprinttype}%
  \clearfield{pages}%
  \clearfield{pagetotal}%
  \clearfield{volume}%
  \clearfield{number}%
  \clearfield{issue}%
  \clearfield{month}%
  \clearfield{day}%
  \clearfield{note}%
  \clearfield{abstract}%
  \clearfield{annotation}%
  \clearlist{location}%
  \clearlist{publisher}%
  \clearfield{address}%
}

% Keep booktitle for conference proceedings
% Keep journaltitle for journal articles
% Keep year
% Keep title


%% Use (and optionally edit if necessary) this .tex if you
%% want an originally numerical reference style like IEEE
%% for your publication list
% \input{pubs-num}

%% sample.bib contains your publications
\addbibresource{sample.bib}

\begin{document}

% \photoR{2.5cm}{test-profile.jpg}

\name{Pouya Ataei}
\tagline{Software Engineer \& Computer Scientist}

\photoR{3cm}{profile-photo.jpg}

% Cropped to square from https://en.wikipedia.org/wiki/Marissa_Mayer#/media/File:Marissa_Mayer_May_2014_(cropped).jpg, CC-BY 2.0
%% You can add multiple photos on the left or right
% \photoR{2.5cm}{mmayer-wikipedia-cc-by-2_0}
% \photoL{2cm}{Yacht_High,Suitcase_High}
\personalinfo{%
  % Not all of these are required!
  % You can add your own with \printinfo{symbol}{detail}
  \email{pouya.ataei.7@gmail.com}
  \linkedin{pouya-ataei-bb1254ba}
  \github{polyhistor} 
  \NewInfoField{researchgate}{\faResearchgate}[https://www.researchgate.net/profile/]
  \researchgate{Pouya-Ataei}
  
%   \orcid{0000-0000-0000-0000} % Obviously making this up too.
  %% You can add your own arbitrary detail with
  %% \printinfo{symbol}{detail}[optional hyperlink prefix]
  % \printinfo{\faPaw}{Hey ho!}
  %% Or you can declare your own field with
  %% \NewInfoFiled{fieldname}{symbol}[optional hyperlink prefix] and use it:
  % \NewInfoField{gitlab}{\faGitlab}[https://gitlab.com/]
  % \gitlab{your_id}
	%%
  %% For services and platforms like Mastodon where there isn't a
  %% straightforward relation between the user ID/nickname and the hyperlink,
  %% you can use \printinfo directly e.g.
  % \printinfo{\faMastodon}{@username@instace}[https://instance.url/@username]
  %% But if you absolutely want to create new dedicated info fields for
  %% such platforms, then use \NewInfoField* with a star:
  % \NewInfoField*{mastodon}{\faMastodon}
  %% then you can use \mastodon, with TWO arguments where the 2nd argument is
  %% the full hyperlink.
  % \mastodon{@username@instance}{https://instance.url/@username}
}

\makecvheader

%% Depending on your tastes, you may want to make fonts of itemize environments slightly smaller
\AtBeginEnvironment{itemize}{\small}

%% Set the left/right column width ratio to 6:4.
\columnratio{0.6}

\cvsection[\faBrain]{Precursor}

\begin{raggedright}
This identification is merely the idea of myself to provide with
specious and precarious sense of permanence. For this idea is
relatively fixed, being based upon carefully selected memories of
my past, which have a preserved and fixed character. Social
conventions therefore, encouraged me to form this idea of myself
with equally abstract and symbolic roles and stereotypes
\end{raggedright}

\vspace{0.5cm}

\begin{paracol}{2}

\cvsection[\faHandHoldingHeart]{What I have to offer?}

\begin{raggedright}
  A passionate software engineer with 10+ years experience developing
  robust code with stress on efficiency, reusability and scalability. I
  have held responsible positions in a number of companies
  specialising in software as a Software Engineer, Tech Lead, Lead Architect, Researcher and Teacher. I constantly perceive the multitudes of complexity within the simplicites and I'm always looking at things through the prism of creativity.
  \end{raggedright}


\switchcolumn

\cvsection[\faHandshake]{What I’m looking for? }

\begin{raggedright}

- Company with a vibrant and positive environment.

- Direct engineering hands-on involvement.

- Time and opportunity to learn and apply emerging technologies.

- Friends to build a good products with.

- Work hard, but not at the weekends.

\end{raggedright}


% \begin{itemize}
%   \item Company with a vibrant and positive environment.
%   \item Direct engineering hands-on involvement.
%   \item Time and opportunity to learn and apply emerging technologies.
%   \item Work hard, but not at the weekends.
% \end{itemize}

\end{paracol}

\vspace{0.5cm}

\cvsection[\faCode]{Skills}

\columnratio{0.5}

\begin{paracol}{2}

\cvskill{Javascript/Typescript}{4.5}
\divider

\cvskill{Golang}{3.5}
\divider

\cvskill{Rust}{3} 
\divider

\cvskill{PHP}{3.5}
\divider

\cvskill{C\#/Java}{3.5}
\divider

\cvskill{C/C++}{3} 
\divider

\cvskill{Software Architecture}{4.5} 
\divider

\cvskill{Kafka/Redpanda}{4} 
\divider

\cvskill{Qualitative/Quantitative Research}{4.5} 
\divider

\cvskill{Distributed Systems}{4} 

\switchcolumn

\cvskill{HTLM5 \& CSS3}{4.5}
\divider

\cvskill{Kubernetes/Docker}{4} 
\divider

\cvskill{Linux/Bash/Awk}{4}
\divider

\cvskill{Node/Bun/Deno}{4.5} 
\divider

\cvskill{SQL/NoSql}{4}
\divider

\cvskill{Python}{4}
\divider

\cvskill{Data Structures \& Algorithms}{4} 
\divider

\cvskill{Design Patterns}{4} 
\divider

\cvskill{Cloud Computing}{4} 
\divider

\cvskill{Mathematics}{4.5} 

\end{paracol}

\vspace{0.5cm}

\columnratio{0.6}



\cvsection[\faBriefcase]{Work Experience}

\cvevent{Lead Development Architect}{Idexx Laboratories}{Sep 2021 -- Current}{Auckland, New Zealand}


\begin{itemize}
\item Stood responsible as one of the company’s lead architects, taking part in critical technological decissions across
different regions 
\item Planned, initiated and implemented a distributed data layer to provide with a delcarative client-driven API.
\item Worked on an event-driven architecture using Kafka, deployed and tested an archetype with consultants from
Confluent 
\end{itemize}

\divider

\cvevent{Tech Lead}{Idexx Laboratories}{Jun 2021 -- Sep 2021}{Auckland, New Zealand}
\begin{itemize}
\item Position Google Maps as the world leader in mobile apps and navigation
\item Oversaw 1000+ engineers and product managers working on Google Maps, Google Places and Google Earth
\end{itemize}

\divider

\cvevent{Vice President of Search Products \& UX}{Google}{2005 --  2010}{Palo Alto, CA}

\divider

\cvevent{Product Manager \& UI Lead}{Google}{Oct 2001 -- July 2005}{Palo Alto, CA}

\begin{itemize}
\item Appointed by the founder Larry Page in 2001 to lead the Product Management and User Interaction teams
\item Optimized Google's homepage and A/B tested every minor detail to increase usability (incl.~spacing between words, color schemes and pixel-by-pixel element alignment)
\end{itemize}

% \divider

% \cvevent{Product Engineer}{Google}{23 June 1999 -- 2001}{Palo Alto, CA}

% \begin{itemize}
% \item Joined the company as employe \#20 and female employee \#1
% \item Developed targeted advertisement in order to use user's search queries and show them related ads
% \end{itemize}

\cvsection{A Day of My Life}


% Adapted from @Jake's answer from http://tex.stackexchange.com/a/82729/226
% \wheelchart{outer radius}{inner radius}{
% comma-separated list of value/text width/color/detail}
% Some ad-hoc tweaking to adjust the labels so that they don't overlap
\hspace*{-1em}  %% quick hack to move the wheelchart a bit left
\wheelchart{1.5cm}{0.5cm}{%
  10/13em/accent!30/Sleeping \& dreaming about work,
  25/9em/accent!60/Public resolving issues with Yahoo!\ investors,
  5/11em/accent!10/\footnotesize\\[1ex]New York \& San Francisco Ballet Jawbone board member,
  20/11em/accent!40/Spending time with family,
  5/8em/accent!20/\footnotesize Business development for Yahoo!\ after the Verizon acquisition,
  30/9em/accent/Showing Yahoo!\ \mbox{employees} that their work has meaning,
  5/8em/accent!20/Baking cupcakes
}

% use ONLY \newpage if you want to force a page break for
% ONLY the currentc column
\newpage

\cvsection{Publications}

\nocite{*}

\printbibliography[heading=pubtype,title={\printinfo{\faBook}{Books}},type=book]

\divider

\printbibliography[heading=pubtype,title={\printinfo{\faFile*[regular]}{Journal Articles}}, type=article]

\divider

\printbibliography[heading=pubtype,title={\printinfo{\faUsers}{Conference Proceedings}},type=inproceedings]







\cvskill{English}{5}
\divider

\cvskill{Spanish}{4}
\divider

\cvskill{German}{3.5} 


\cvsection{Education}

\cvevent{M.S.\ in Computer Science}{Stanford University}{Sept 1997 -- June 1999}{}

\divider

\cvevent{B.S.\ in Symbolic Systems}{Stanford University}{Sept 1993 -- June 1997}{}

\newpage

\cvsection{Referees}

% \cvref{name}{email}{mailing address}
\cvref{Prof.\ Alpha Beta}{Institute}{a.beta@university.edu}
{Address Line 1\\Address line 2}

\divider

\cvref{Prof.\ Gamma Delta}{Institute}{g.delta@university.edu}
{Address Line 1\\Address line 2}



\end{document}
