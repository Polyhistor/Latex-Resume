%%%%%%%%%%%%%%%%%
% This is an example CV created using altacv.cls (v1.6.4, 13 Nov 2021) written by
% LianTze Lim (liantze@gmail.com), based on the
% Cv created by BusinessInsider at http://www.businessinsider.my/a-sample-resume-for-marissa-mayer-2016-7/?r=US&IR=T
%
%% It may be distributed and/or modified under the
%% conditions of the LaTeX Project Public License, either version 1.3
%% of this license or (at your option) any later version.
%% The latest version of this license is in
%%    http://www.latex-project.org/lppl.txt
%% and version 1.3 or later is part of all distributions of LaTeX
%% version 2003/12/01 or later.
%%%%%%%%%%%%%%%%

%% Use the "normalphoto" option if you want a normal photo instead of cropped to a circle
% \documentclass[10pt,a4paper,normalphoto]{altacv}

\documentclass[10pt,a4paper,ragged2e,withhyper]{altacv}
\usepackage[none]{hyphenat}




%% AltaCV uses the fontawesome5 package.
%% See http://texdoc.net/pkg/fontawesome5 for full list of symbols.

% Change the page layout if you need to
\geometry{left=1.25cm,right=1.25cm,top=1.5cm,bottom=1.5cm,columnsep=1.2cm}

% The paracol package lets you typeset columns of text in parallel
\usepackage{paracol}
\usepackage{graphicx}
\usepackage{fontawesome5}
% \usepackage{biblatex}
% \addbibresource{sample.bib}



\renewcommand{\cvevent}[4]{%
  \textbf{#1} % Job title
  \hfill % This will push the right minipage to the right
  \begin{minipage}[t]{.5\linewidth}
    \raggedleft % Aligns content to the right
    \small#3 % Location
    \\ % New line for the date right below location
    #4 % Date
  \end{minipage}
  \vspace{\baselineskip} % Adds some space after the event
}





\newlength{\mylen}

\makeatletter
% Modify the cvsection command to accept an icon name as the first argument
\RenewDocumentCommand{\cvsection}{o m}{%
  % Check if an icon name is provided
  \IfValueTF{#1}{%
    % If an icon is provided, measure the width including the icon
    \settowidth{\mylen}{\Large\bfseries\color{heading}#1\hspace{0.0em}#2}
    % Add space before the section title
    \vspace{1.5ex}
    % Print the section title with icon
    {\Large\bfseries\color{heading}#1\hspace{0.2em}#2}\\[-1ex]
  }{%
    % If no icon is provided, proceed as before
    \settowidth{\mylen}{\Large\bfseries\color{heading}#2}
    % Add space before the section title
    \vspace{1.5ex}
    % Print the section title without icon
    {\Large\bfseries\color{heading}#2}\\[-1ex]
  }
  % Center the rule: adjust the length here if you want to change the thickness
  \makebox[\mylen]{\rule{\mylen}{0.0pt}} % Use the measured width for the rule
  \vspace{2ex} % Adjusted space after the title underline
}
\makeatother

% Change the font if you want to, depending on whether
% you're using pdflatex or xelatex/lualatex
\ifxetexorluatex
  % If using xelatex or lualatex:
  \setmainfont{Lato}
\else
  % If using pdflatex:
  \usepackage[default]{lato}
\fi

% Change the colours if you want to
\definecolor{VividPurple}{HTML}{3E0097}
\definecolor{SlateGrey}{HTML}{2E2E2E}
\definecolor{LightGrey}{HTML}{666666}
% \colorlet{name}{black}
% \colorlet{tagline}{PastelRed}
\colorlet{heading}{VividPurple}
\colorlet{headingrule}{VividPurple}
% \colorlet{subheading}{PastelRed}
\colorlet{accent}{VividPurple}
\colorlet{emphasis}{SlateGrey}
\colorlet{body}{LightGrey}

% Change some fonts, if necessary
% \renewcommand{\namefont}{\Huge\rmfamily\bfseries}
% \renewcommand{\personalinfofont}{\footnotesize}
% \renewcommand{\cvsectionfont}{\LARGE\rmfamily\bfseries}
% \renewcommand{\cvsubsectionfont}{\large\bfseries}

% Define the length outside the command, to ensure it's only defined once






% Change the bullets for itemize and rating marker
% for \cvskill if you want to
\renewcommand{\itemmarker}{{\small\textbullet}}
\renewcommand{\ratingmarker}{\faCircle}

%% Use (and optionally edit if necessary) this .tex if you
%% want to use an author-year reference style like APA(6)
%% for your publication list
% Custom simplified bibliography for resume - shows title, year, publication only
\usepackage[backend=biber,style=apa6,sorting=ydnt]{biblatex}
\defbibheading{pubtype}{\cvsubsection{#1}}
\renewcommand{\bibsetup}{\vspace*{-\baselineskip}}
\AtEveryBibitem{\makebox[\bibhang][l]{\itemmarker}}
\setlength{\bibitemsep}{0.25\baselineskip}
\setlength{\bibhang}{1.25em}

% Hide author names
\AtEveryBibitem{%
  \clearname{author}%
  \clearname{editor}%
  \clearname{translator}%
}

% Hide unnecessary fields for cleaner display
\AtEveryBibitem{%
  \clearfield{doi}%
  \clearfield{isbn}%
  \clearfield{issn}%
  \clearfield{url}%
  \clearfield{urldate}%
  \clearfield{eprint}%
  \clearfield{eprinttype}%
  \clearfield{pages}%
  \clearfield{pagetotal}%
  \clearfield{volume}%
  \clearfield{number}%
  \clearfield{issue}%
  \clearfield{month}%
  \clearfield{day}%
  \clearfield{note}%
  \clearfield{abstract}%
  \clearfield{annotation}%
  \clearlist{location}%
  \clearlist{publisher}%
  \clearfield{address}%
}

% Keep booktitle for conference proceedings
% Keep journaltitle for journal articles
% Keep year
% Keep title


%% Use (and optionally edit if necessary) this .tex if you
%% want an originally numerical reference style like IEEE
%% for your publication list
% \input{pubs-num}

%% sample.bib contains your publications
\addbibresource{sample.bib}

\begin{document}

% \photoR{2.5cm}{test-profile.jpg}

\name{Dr. Pouya Ataei}
\tagline{Big Data \& AI Engineer | Computer Scientist | PhD in Big Data }

\photoR{3cm}{profile-photo.png}

% Cropped to square from https://en.wikipedia.org/wiki/Marissa_Mayer#/media/File:Marissa_Mayer_May_2014_(cropped).jpg, CC-BY 2.0
%% You can add multiple photos on the left or right
% \photoR{2.5cm}{mmayer-wikipedia-cc-by-2_0}
% \photoL{2cm}{Yacht_High,Suitcase_High}
\personalinfo{%
  % Not all of these are required!
  % You can add your own with \printinfo{symbol}{detail}
  \email{pouya.ataei.7@gmail.com}
  \linkedin{pouya-ataei-bb1254ba}
  \github{polyhistor} 
  \NewInfoField{researchgate}{\faResearchgate}[https://www.researchgate.net/profile/]
  \researchgate{Pouya-Ataei}
  
%   \orcid{0000-0000-0000-0000} % Obviously making this up too.
  %% You can add your own arbitrary detail with
  %% \printinfo{symbol}{detail}[optional hyperlink prefix]
  % \printinfo{\faPaw}{Hey ho!}
  %% Or you can declare your own field with
  %% \NewInfoFiled{fieldname}{symbol}[optional hyperlink prefix] and use it:
  % \NewInfoField{gitlab}{\faGitlab}[https://gitlab.com/]
  % \gitlab{your_id}
	%%
  %% For services and platforms like Mastodon where there isn't a
  %% straightforward relation between the user ID/nickname and the hyperlink,
  %% you can use \printinfo directly e.g.
  % \printinfo{\faMastodon}{@username@instace}[https://instance.url/@username]
  %% But if you absolutely want to create new dedicated info fields for
  %% such platforms, then use \NewInfoField* with a star:
  % \NewInfoField*{mastodon}{\faMastodon}
  %% then you can use \mastodon, with TWO arguments where the 2nd argument is
  %% the full hyperlink.
  % \mastodon{@username@instance}{https://instance.url/@username}
}

\makecvheader

%% Depending on your tastes, you may want to make fonts of itemize environments slightly smaller
\AtBeginEnvironment{itemize}{\small}

%% Set the left/right column width ratio to 6:4.
\columnratio{0.6}

% \cvsection[\faBrain]{Precursor}

% \begin{raggedright}
% This identification is merely the idea of myself to provide with
% specious and precarious sense of permanence. For this idea is
% relatively fixed, being based upon carefully selected memories of
% my past, which have a preserved and fixed character. Social
% conventions therefore, encouraged me to form this idea of myself
% with equally abstract symbolic roles, and stereotypes. 
% \end{raggedright}

% \vspace{0.5cm}

% \begin{paracol}{2}

\cvsection[\faHandHoldingHeart]{Contribution to Australia's National Interest}

\begin{raggedright}
I currently architect critical data infrastructure for Australia's National Electricity Market at Vector Technology Solutions. This infrastructure processes telemetry for Blue Current—a sovereign asset co-owned by the Queensland Investment Corporation (QIC)—serving 2.7 million smart meters. The serverless architecture I contribute to enables the grid stability required for Australia's Net Zero transition. Simultaneously, I develop machine learning solutions for infrastructure mapping (PSPS) in California, transferring global wildfire mitigation best practices to Australia.

  \vspace{0.5cm}
With a PhD in Big Data (2024), I address the high failure rate of data projects through resilient distributed systems. My reference architecture, Terramycelium (2025), demonstrates processing of 771,305 messages with 0.0000148s latency, supporting Australia's low-emission technology goals.
\end{raggedright}

\vspace{0.4cm}

\cvsection[\faChartLine]{Academic Impact \& Research Recognition}

\begin{raggedright}
\textbf{Google Scholar Metrics} (as of December 2025)

\vspace{0.4cm}
\begin{tabular}{@{}ll@{}}
\textbf{h-index:} & 8 \\[0.15cm]
\textbf{Total Citations:} & 173+ \\[0.15cm]
\textbf{i10-index:} & 8 \\[0.15cm]
\textbf{Publications:} & 17 peer-reviewed (journals, conferences, book) \\
\end{tabular}

\vspace{0.5cm}
\textbf{Publication Venues:}

\vspace{0.2cm}
IEEE Access (IF: 3.4) • Journal of Big Data (IF: 8.1) • Frontiers in Big Data (IF: 3.1) 

\vspace{0.1cm}
Asia-Pacific Software Engineering Conference (IEEE) • American Conference on Information Systems
\end{raggedright}



\vspace{0.5cm}

\cvsection[\faCode]{Skills}

\columnratio{0.5}

\begin{paracol}{2}
  \cvskill{Big Data Systems}{4.5}
  \divider
  \cvskill{Machine Learning / Deep Learning}{4}
  \divider
  \cvskill{TensorFlow / PyTorch}{3.5}
  \divider
  \cvskill{Data Engineering}{4}
  \divider
  \cvskill{Research Methods (Qual/Quant)}{4.5}
  \divider
  \cvskill{Distributed Systems}{4}
  \divider
  \cvskill{Software Architecture}{4.5}
  \divider
  \cvskill{Python}{4.5}
  \divider
  \cvskill{Golang}{3.5}
  \divider
  \cvskill{Rust}{3}
  \divider

  \switchcolumn
  \cvskill{Apache Kafka / Stream Processing}{4}
  \divider
  \cvskill{AWS, Google Cloud, Azure}{4}
  \divider
  \cvskill{Kubernetes / Docker}{4}
  \divider
  \cvskill{Terraform / Infrastructure as Code}{4}
  \divider
  \cvskill{SQL / NoSQL Databases}{4}
  \divider
  \cvskill{Vector Databases / Elasticsearch}{4}
  \divider
  \cvskill{Data Pipelines / ETL}{4}
  \divider
  \cvskill{Statistical Analysis}{4}
  \divider
  \cvskill{Linux / Bash}{4}
  \divider
  \cvskill{Javascript / Typescript}{4.5}
  \divider
  \end{paracol}

\vspace{0.5cm}

\columnratio{0.6}


\makeatletter
% Define or renew the cvevent command
\RenewDocumentCommand{\cvevent}{m m m m}{%
  % Arguments: #1 = Position, #2 = Company, #3 = Date Range, #4 = Location
  \vspace{1.5ex} % Space before the event
  {\bfseries\large #1} at {\bfseries\large #2} % Position at Company
  \hfill % Fill space
  {\small\itshape\faCalendar #3} % Date range
  \\ % New line
  {\small\itshape #4} % Location
  \par % End of paragraph
  \vspace{1ex} % Space after the event before details begin
}
\makeatother

% \vspace{1cm}

\cvsection[\faBriefcase]{Work Experience}


\cvevent{Principal Data Engineer}{Vector Limited}{May 2025 -- Current}{Auckland, New Zealand}

\begin{itemize}
  \item Architecting critical serverless infrastructure for Blue Current (National Electricity Market partner), processing 2 billion daily data intervals from 2.7 million smart meters to support grid stability.
  \item Developing machine learning models for Proactive Power Safety Shutoff (PSPS) mapping in California (San Diego), applying adaptive algorithms to optimize grid resilience and wildfire mitigation.
  \item Drive AWS cloud cost efficiency through best practices in Glue, Athena, S3, DynamoDB, and event-driven architectures.
  \item Implement automated data validation, lineage tracking, and observability to maintain data quality and ensure compliance with regulatory frameworks.
  \item Design and scale Diverge's data platform to support both real-time and batch energy analytics with a focus on performance, security, and cost-efficiency.
  \item Architect and deploy energy data algorithms while ensuring efficient and scalable model inference integrated with Diverge's services.
  \item Lead test automation strategies that embed data quality checks into CI/CD pipelines and foster an architectural culture that promotes iterative, autonomous development.
  \item Mentor junior engineers, fostering a high-performance engineering culture aligned with the company's long-term technology vision.
\end{itemize}

\vspace{0.5cm}

\textit{Tech \& Tools: AWS (Lambda, DynamoDB, S3, Glue, Athena), Python, Serverless Architecture, Data Engineering, Machine Learning, CI/CD, Event-Driven Architecture, Data Science}

\divider

\cvevent{Engineering Manager}{Invenco by GVR}{Sep 2024 -- May 2025}{Auckland, New Zealand}

\begin{itemize}
  \item Led the 'Engage' media platform, a high-throughput distributed system delivering real-time advertising content to a nationwide network of payment terminals across the US and APAC markets.
  \item Serve as the principal architect for the entire 'Engage' domain, holding sole responsibility for architectural decisions and technical direction across all systems and services.
  \item Design and implement event-driven architectures using Kafka and AWS EventBridge, enabling real-time data processing and system integration at scale.
  \item Establish and enforce engineering excellence through comprehensive TypeScript guidelines, Clean Architecture principles, and Domain-Driven Design practices across all teams.
  \item Implement Value Stream Management and Flow Framework methodologies to optimize delivery pipelines and enhance team productivity while reducing cognitive load.
  \item Architect and implement advanced observability solutions using OpenTelemetry, significantly enhancing system monitoring and performance tracking capabilities.
  \item Drive technical excellence through cloud-native architectures, leveraging AWS, Kubernetes, and Kafka for robust distributed systems.
  \item Oversee development across diverse technology stacks including Node.js, Go, C\#/.NET, and Snowflake for data warehousing solutions.
  \item Guide architectural decisions and technology choices to support the organization's growth and scalability goals.
  \end{itemize}

\vspace{0.5cm}

\textit{Tech \& Tools: AWS, OpenTelemetry, Node.js, Kubernetes, Docker, Terraform, GitLab CI/CD, Microservices, Python, Bash, Vector Databases, TensorFlow}

\divider

\cvevent{Development Team Leader - Cloud Architecture}{Invenco by GVR}{July 2024 -- Sep 2024}{Auckland, New Zealand}

\begin{itemize}
\item Lead a cross-functional team in developing and maintaining cloud-based payment solutions, fostering innovation and continuous improvement.
\item Architect and implement scalable cloud services, ensuring high performance and security standards are met.
\item Created and implemented a comprehensive observability plan using OpenTelemetry, significantly enhancing system monitoring and troubleshooting capabilities.
\item Successfully scaled our terminal infrastructure to handle over 300,000 requests in under 30 minutes, dramatically improving system capacity and responsiveness.
\item Drive adoption of modern software engineering practices, including microservices architecture and serverless computing.
\item Collaborate with stakeholders to define technical roadmaps, balancing immediate needs with long-term strategic goals.
\item Mentor team members, facilitating their professional growth and fostering a collaborative work environment.
\item Spearhead the integration of DevOps practices, improving deployment frequency and reducing time-to-market for new features.
\item Stay abreast of emerging technologies, evaluating their potential impact on the organization's cloud strategy.
\end{itemize}

\vspace{0.5cm}

\textit{Tech \& Tools: AWS, Kubernetes, Kafka, OpenTelemetry, Node.js, Go, C\#/.NET, Snowflake, Terraform, Docker, GitLab CI/CD, Python, Bash, Vector Databases, TensorFlow}

\divider

% \vspace{0.5cm}

\cvevent{Lead Development Architect}{Idexx Laboratories}{Sep 2021 -- July 2024}{Auckland, New Zealand}


\begin{itemize}
\item Stood responsible as one of the company's lead architects, taking part in critical technological decisions across
different regions.
\item Worked on an event-driven architecture using Kafka, deployed and tested an archetype with consultants from Confluent. 
\item Presented the company and engaged with representatives from Databricks, Heap, Apollo, AWS, Gitlab, Confluent, BizzDesign and others.
\item Created a scientific methodology for software delivery. 
\item Introduced novel data-driven approaches to decision making. 
\item Designed infrastructure, applications and integrations on daily basis.
\item Coached and mentored junior, intermediate and senior engineers, providing guidance on best practices, and best ways to learn. 
\item Supported various teams in various feature development through high-level models and also hands-on coding.
\item Completed various leadership courses, including CASCADE, and practiced servant leadership principles, prioritizing team growth and well-being.
\item Applied concepts from "Leaders Eat Last" and "The Servant" to foster a culture of trust and collaboration, leveraging Team Topologies and Domain-Driven Design to streamline communication and align team efforts.
\end{itemize}

\vspace{0.5cm}

\textit{Tech \& Tools: Archimate, TOGAF, ADM, BDD, DDD, AWS, Kafka, Go, Graph, SQL ,Terraform, Kubernetes, Elasticsearch, Typescript}

\divider


\cvevent{Tech Lead}{Idexx Laboratories}{Jun 2021 -- Sep 2021}{Auckland, New Zealand}

\begin{itemize}
  \item Served as a key lead architect, driving critical technological decisions and strategies across various regions, emphasizing my pivotal role in shaping the company's technological landscape.
  \item Designed and deployed a distributed data layer, introducing a declarative client-driven API to enhance system flexibility and client interaction efficiency.
  \item Engineered an event-driven architecture leveraging Kafka, including the deployment and rigorous testing of a prototype in collaboration with Confluent consultants, showcasing my expertise in modern, scalable system design.
  \item Actively engaged in feature development, technical debt reduction, and incident management, contributing to continuous improvement in system reliability and development practices.
  \item Presented scientific approaches and innovative solutions across different teams to address existing challenges and develop new features, fostering a culture of continuous learning and cross-team collaboration.
\end{itemize}



\vspace{0.5cm}

\textit{Tech \& Tools: Laravel, Terraform, AWS, Apollo, Go, Rust, Node, React, PHP, SQL, Kafka, Python, Kubernetes, Elasticsearch, Typescript}

\divider

\cvevent{Senior Fullstack Developer}{ezyVet}{Jul 2020 -- Jun 2021}{Auckland, New Zealand}


\begin{itemize}
  \item Elevated from Senior Frontend Developer to Frontend Tech Lead within Ezyvet, taking comprehensive responsibility for all frontend applications, demonstrating rapid professional growth and trust in my leadership and technical capabilities.
  \item Championed the integration of best practices and cutting-edge technologies into the team's workflow, significantly enhancing project outcomes and efficiency. Actively contributed to the development and refactoring of complex features, reinforcing the team's adaptability and technical prowess.
  \item Orchestrated a major refactor of the frontend codebase, quintupling its modifiability, performance, and security, which substantially boosted the application's reliability and user satisfaction.
  \item Spearheaded the recruitment process for new talent, playing a pivotal role in interviewing, onboarding, and mentoring new hires, thereby fortifying the team's capabilities and fostering a culture of continuous learning and improvement.
  \item Addressed critical engineering challenges in both the frontend and platform domains, implementing solutions that dramatically optimized our deployment pipeline—cutting build and deployment times from an average of 45 minutes to just 8 minutes, greatly accelerating product iteration cycles and enhancing team productivity.
\end{itemize}


  \vspace{0.5cm}

  \textit{Tech \& Tools: React, Typescript, AWS, Go, Laravel, Kubernetes, Terraform, PHP, Node}

  \divider



\cvevent{Fullstack Developer}{Infosys}{Feb 2020 -- Jul 2020}{Auckland, New Zealand}


\begin{itemize}
  \item Deployed to Vodafone to take part in the development of a new E-shop.
  \item Began as a frontend developer, but I rapidly expanded my role to encompass CMS and backend development, demonstrating flexibility and a broad skill set.
  \item Was one of the main engineers that worked on the frontend and the e-commerce backend. The work included checkout processes, cart validation, product pages, and component design, along with optimizing SEO, implementing analytics, conducting UI and end-to-end testing, and driving integration projects.
  \item Worked on various legacy Oracle systems and written integration and anti-corruption layers.
  \item Collaborated closely with a diverse range of stakeholders on a daily basis, focusing on the development of innovative concepts, and the identification and resolution of bugs and defects, ensuring continual improvement and operational excellence.
\end{itemize}




\vspace{0.5cm}

\textit{Tech \& Tools: React, Flask, DynamoDB, Jenkins, AWS, BigCommerce, Java, Python, Javascript}


\divider



\cvevent{Frontend Engineer}{Navigate Travel}{Jun 2019-- Feb 2020}{Auckland, New Zealand}


\begin{itemize}
  \item Spearheaded the migration of three websites from WordPress to Gatsby, integrating React, Contentful, and Node (NestJS), successfully transitioning 900 blog posts with a keen focus on SEO optimizations and schema markups.
  \item Enhanced the performance and SEO of existing WordPress sites, implementing advanced optimization strategies to improve search engine rankings and user engagement.
  \item Developed a cross-browser, responsive web application that combines dynamic and static elements, delivering an exceptional user experience on all devices.
  \item Managed API concurrency issues, dynamically updating product prices, names, and descriptions to ensure real-time accuracy and responsiveness.
  \item Contributed to refining the company's SEO strategy, leading to measurable improvements in web visibility and traffic.
  \item Collaborated in the UI/UX design process, creating intuitive and user-friendly interfaces that enhance customer satisfaction and engagement.
  \item Leveraged emerging technologies in both front-end and back-end development, significantly increasing application performance and speed.
  \item Implemented a polyglot persistence architecture, enhancing system efficiency and scalability to support growing data demands and diverse data types.
\end{itemize}


\vspace{0.5cm}

\textit{Tech \& Tools: Gatsby, Contentful, React, Node, Sass, Graphql, Netlify, Vercel, Gitlab Pipelines, AWS, Google Cloud}

 

\divider

\cvevent{Founder, Lead Fullstack Developer}{Pouyaraveshan Academy}{Feb 2017-- June 2019}{Tehran, Iran}

\begin{itemize}
  \item Implemented servant leadership principles from "The Servant" to empower and uplift team members, creating a collaborative and growth-oriented environment, drawing from insights in the book "Leaders Eat Last".
  \item Optimized SEO strategies, achieving an average website position of 8.5 and a click-through rate (CTR) of 6.5\%, significantly enhancing online visibility.
  \item Directed a team to launch an early version of our website's Accelerated Mobile Pages (AMP), resulting in a 70\% increase in traffic.
  \item Conducted computer science and software engineering workshops, covering a wide array of technologies and empowering participants with new skills.
  \item Oversaw key operations including marketing, content creation, administration, student engagement, system development, and orchestration, driving comprehensive organizational growth.
  \item Regularly developed, maintained, and enhanced pouyaraveshan.org, ensuring an engaging user interface and robust backend functionality.
  \item Implemented and fine-tuned a Learning Management System (LMS), both improving client experience and streamlining organizational processes through automation.
  \item Implemented an online learning system using Adobe Connect and custom built backends. I have then integrated this system with our LMS, and POS systems. 
\end{itemize}



\vspace{0.5cm}

\textit{Tech \& Tools: ReactJS, NodeJs, Mongo, Adobe Connect, Mysql, PHP, Golang, C\#, .Net Core, Java}

\divider

\cvevent{Full Stack Developer}{Cake Creative Digital Lab}{June 2015-- Nov 2016}{Kuala Lumpur, Malaysia}

\begin{itemize}
  \item Engaged in diverse web development projects, including front-end design and e-commerce, enhancing system customization and exclusivity.
  \item Reduced rework by 22\% and cut costs by 15\% by engineering a reusable utilities library.
  \item Achieved high user experience scores across all web development projects, ensuring an intuitive and effective user interface.
  \item Optimized project performance to consistently meet or exceed a 95\% Google Pagespeed score.
  \item Conceptualized and prototyped an average of 25 new product features annually, significantly contributing to product enhancement.
  \item Led the development of a website that garnered the 2015 Malaysia Website Award for outstanding UI/UX and structure.
  \item Played a key role in the full project lifecycle, from development and maintenance to deployment, emphasizing robust and scalable solutions.
  \item Collaborated with the development team to ascertain user requirements, formulate use cases, prioritize tasks, and draft comprehensive documentation.
  \item Conducted thorough analyses of web-based systems to identify and rectify defects, bolster performance, and fortify security measures.
\end{itemize}

\vspace{0.5cm}

\textit{Tech \& Tools: Wordpress, Magento, Opencart, AngularJS, Vanilla JS, PHP, C\#, .Net Core, Golang}

\divider

\cvevent{Workshop Assistant, Python Developer }{Kidocode}{Feb 2015-- May 2015}{Kuala Lumpur, Malaysia}


\begin{itemize}
  \item Completed a 3-month internship, during which I facilitated Python, NoSQL, Cloud, and Web Development workshops at various universities in Kuala Lumpur, enhancing educational outreach and technology adoption.
  \item Diagnosed and resolved student's coding issues during workshop sessions, ensuring smooth and effective learning experiences.
  \item Played a role in the development of the company's loyalty campaign platform, coding various aspects from database to frontend.
  \item Engaged in youth education initiatives, teaching programming fundamentals through both hands-on (plugged) and theoretical (unplugged) methods, fostering early interest in technology.
\end{itemize}


\vspace{0.5cm}

\textit{Tech \& Tools: Django, NumPy, Matplotlib, AngularJS, Mongo, Python}


\vspace{0.5cm}



\renewcommand{\cvevent}[4]{%
  \noindent
  \begin{tabular}{@{}p{0.4\linewidth}p{0.28\linewidth}p{0.2433\linewidth}@{}}
    \small\textbf{#1} & % Degree
    \small{#2} & % Location
    \raggedleft\arraybackslash\small{#3} % Date (right-aligned)
  \end{tabular}
  \par % End the tabular line and provide some space before the next line
  \vspace{0.5em} % Adjust the vertical space after the event
}







\cvsection[\faGraduationCap]{Education}

\cvevent{B.S.C\ in Software Engineering (Dual Degree)}{Staffordshire University}{Jan 2012 -- April 2015}{}

\divider

\cvevent{B.S.C\ in Software Engineering (Dual Degree)}{Asia Pacific University of Technology}{Jan 2012 -- April 2015}{}

\divider

\cvevent{M.S.C\ in Software Engineering}{Staffordshire University}{September 2015 -- Feb 2017}{}

\divider

\cvevent{PhD in Computer Science}{Auckland University of Technology}{Feb 2019 -- September 2024}{}

\divider

\cvevent{Nano Degree\ in Data Architecture}{Udacity}{Dec 2022 -- Feb 2023}{}



\vspace{0.5cm}

\cvsection[\faCertificate]{Certificates}

\cvevent{PyResearcher - Sixty Hours (Python - MongoDB - NumPy - Matplotlib - Pandas)}{Pycademy}{April 2015}{}

\divider

\cvevent{MCPS: Microsoft Certified Professional}{Microsoft}{October 2015}{}

\divider

\cvevent{MS: Programming in HTML5 with JavaScript and CSS3}{Microsoft}{October 2015}{}

\divider

\cvevent{Data Streaming Engineer}{Confluent}{December 2025}{}



\vspace{0.5cm}

\cvsection[\faFlask]{Publications}



\nocite{*}

  \printbibliography[heading=pubtype,title={\printinfo{\faBook}{Books}},type=book]

  \divider

 \printbibliography[heading=pubtype,title={\printinfo{\faFile*[regular]}{Journal Articles}}, type=article]

 \divider

\printbibliography[heading=pubtype,title={\printinfo{\faUsers}{Conference Proceedings}},type=inproceedings]


\vspace{0.5cm}

\cvsection[\faTrophy]{International Recognition \& Awards}

\begin{itemize}
  \item \textbf{2015 Malaysia Website Award}\\
        Outstanding UI/UX and Structure \hfill 2015
  \item \textbf{Confluent Certified Data Streaming Engineer}\\
        Industry certification in distributed data systems \hfill December 2025
\end{itemize}

\vspace{3mm}

\cvsection[\faAtom]{Professional Memberships}

\begin{itemize}
  \item \textbf{IEEE Computer Society}\\
        Academic Contributor \hfill Jan 2021 - Present
  \item \textbf{Association for Information Systems (AIS)}\\
        Academic Contributor \hfill May 2019 - Present
\end{itemize}

\vspace{5mm} % Adjusted for better spacing

\cvsection[\faFilePowerpoint]{Presentations Delivered}

\begin{itemize}
 

  \item International Conference on Control, Electronics and Computer Technology - ICCEIT 2017, India \hfill 2017
  \item International Conference on Control, Electronics, and Computer Technology - ICCECT 2017, Malaysia \hfill 2017
  \item Australasian Conference on Information Systems - ACIS, New Zealand \hfill 2019
  \item American Conference on Information Systems - AMCIS, USA \hfill 2020
  \item Asia-Pacific Software Engineering Conference - APSEC, Taiwan \hfill 2021
  \item Kafka and Terraform Meetup - Confluent, New Zealand \hfill 2023 
  \item Geekle- World Global React Conference \hfill Multiple (Latest in 2023)
  \item American Conference on Information Systems - AMCIS, USA \hfill 2024

\end{itemize}











\end{document}
